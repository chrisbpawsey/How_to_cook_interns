\section{Creating Examples on getexample}

The examples included in the getexample library were created by internship students at Pawsey with their supervisor while the source codes within the
getexample were obtained from some wiki pages and websites and were acknowledged in the examples. Some of the examples were also created due to the
needs of research students for their internship projects at Pawsey. The examples obtained in the getexample are mainly for teaching the users how to work
with parallel programming including MPI, OpenMP and hybrid jobs such as OpenMP/MPI which utilise basic source codes similar to "Hello world" program and
submit these jobs to different supercomputers such as Magnus, Zeus and Zythos using different program environments and compiler modules. 

Each individual example on the getexample occupies a directory that is reserved for them and each example consists of three files listed below:

\begin{itemize}    
\item SLURM (Simple Linux Utility for Resource Management) : This allows the users to submit batch jobs to the supercomputers, check on their status and 
cancel them if needed. It contains the necessary detail about the name of the executable, the results directory, the name of the output file and 
the jobID. It also allows the users to edit how many nodes the code requires to run on the HPC systems, the duration of the task that it takes, which 
partition to be used and their account name. The SLURM initially creates a scracth directory for the example to run in and the results are outputted to 
a log file. Then, it creates a group directory in which the results directory is located for that example. Once the output file is completed within the 
scratch, the output file is then moved to the results directory located in the group directory and the scratch directory then gets removed.
\item Source Code : This is usually a source code in c or Fortran and taken from wiki pages to run the example.
\item README : This file is an executable bash script which can be read and run by the users. It provides details about what the source code does,
how to compile the source code depending on the program environment such as Cray, Intel, GNU or PGI, what can be modified in the SLURM directives, and 
a set of instruction on how to submit the SLURM to the supercomputers including which specific commands to use for particular supercomputers. It can be
executed by simply typing ./README which then compiles the source code and submits the batch job to the chosen supercomputer.
\end{itemize}

For the examples in the getexample tool to be user friendly, helpful and efficient for working with HPC resources, there are many design considerations 
to be held as listed below:

\begin{itemize}
\item Before introducing the examples to the users, it should be made sure that the examples run without encountering any errors. For example,
they should be suitable for the current operating systems with the use of correct commands for the different compiler modules such as Intel, GNU,
and PGI or different program environments such as Cray.
\item All the files to run the example should be included with the example.
\item To minimise confusion on how to perform the example on the supercomputers, the example should be executed with a single command. For example, 
if a simple shell script is used, it should be run as ./README.
\item The examples should have enough instructions about what each command does and which parts of the SLURM and the README file can be modified so that
all the files and the scripts should be able to edited easily by the users for their preferences.
\item It must be ensured that the README and SLURM have enough information about how to change certain parts of these files so that the users can run the 
example how they wish to on the supercomputers. These instructions should be understandable by the new users who are not very experienced with these
systems. For example, if the user wishes to use more nodes on the supercomputers, one should be able to know where to change it from.
\end{itemize}
 
Even though, the examples in the getexample tool are designed in a such way that once they are downloaded, the user should be able to run them without
having to modify them. However, this is not always the case as some of the supercomputers in Pawsey Supercomputing Centre such as Zythos due to
different set ups require the account name which is customized for each user and without the correct account name in the SLURM, the code fails to run. 
Therefore, the examples in the getexample tool ensures that the users are informed on how to change their account name located in the SLURM file.
Furthermore, they assist the users on how to change number of nodes used within the supercomputers and increase or reduce the number of cores used


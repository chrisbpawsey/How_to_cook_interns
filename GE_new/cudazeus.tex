\subsubsection{CUDA Examples}

The CUDA programming is a heterogeneous model where both CPU and GPU nodes are used. In CUDA, the host refers to the CPU and its memory, whereas the 
device refers to the GPU an its memory. Therefore, a CUDA code running on the host have access to the memory on the host as well as the device. It also 
executes kernel functions on the device which are executed by GPU threads in parallel. A basic CUDA works by declaring and allocating host and device 
memory. Then, it initializes the host data and transfers the data from the host to the device. Once it executes the kernel function, it transfers the 
results from the device to the host.

The getexample tool includes a basic \emph{"Hello world"} CUDA code, hello\_cuda.cu for Zeus. It uses 1 node with any generic GPU card, and this was 
defined in the SLURM file as:

\begin{tcolorbox}
\begin{Verbatim}[fontsize=\scriptsize]
#SBATCH --partition=workq
#SBATCH --nodes=1
#SBATCH --gres=gpu:1
\begin{Verbatim}
\begin{tcolorbox}

To compile the CUDA code correctly, the cuda module should be loaded and this was done in both README and the SLURM file by using the following line:

\begin{tcolorbox}
\begin{Verbatim}[fontsize=\scriptsize]
module load cuda
\end{Verbatim}
\end{tcolorbox}

Then, to submit this task to Zeus, the following command was used:

\begin{tcolorbox}
\begin{Verbatim}[fontsize=\scriptsize]
./$EXECUTABLE >> ${OUTPUT}
\end{Verbatim}
\end{tcolorbox}

The CUDA code was compiled by using the following compiler line in the README file:

\begin{tcolorbox}
\begin{Verbatim}[fontsize=\scriptsize]
nvcc hello_cuda.cu -o hello_cuda_gnu
\end{Verbatim}
\end{tcolorbox}

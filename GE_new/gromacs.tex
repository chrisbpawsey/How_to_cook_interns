\subsubsection{GROMACS}

GROMACS is also known as GROningen MAchine for Chemical Simulations which is a molecular dynamics package. It is designed for simulations of proteins, 
lipids, and nucleic acids. It contains a script to convert molecular coordinates from Protein Data Bank (PDB) files into formats. The simulation starts 
running when a configuration file for the simulation of the molecules is created. This produces a trajectory file which then can be analysed with various 
tools.

GROMACS can run on Central Processing Units known as CPUs as well as Graphics Processing Units called as GPUs. It can be executed in parallel, using 
Message Passing Interface (MPI) or threads. For the \emph{getexample} tool, GROMACS example was run utilizing MPI similar to LAMMPS example, but it was 
more complicated to implement.

This example requested 2 nodes with 24 MPI tasks per node with a time allocation of 40 minutes, as it is a relatively big task to run. These changes
were shown in the SLURM directives as:

\begin{tcolorbox}
\begin{Verbatim}[fontsize=\scriptsize]
#!/bin/bash -l
#SBATCH --job-name=hostname
#SBATCH --partition=debugq
#SBATCH --nodes=2
#SBATCH --time=00:40:00
#SBATCH --export=NONE
\end{Verbatim}
\end{tcolorbox}

This code was also run on the GNU environment and hence, it was necessary to change the program environment from Cray to the GNU and load the gromacs
module. This module was needed to compile the source code and access the original gromacs data set.

\begin{tcolorbox}
\begin{Verbatim}[fontsize=\scriptsize]
module swap PrgEnv-cray PrgEnv-gnu
module load gromacs/5.1.1
\end{Verbatim}
\end{tcolorbox}

The generic variables for this job were defined in the SLURM just like the other examples, but with the addition of the input and GROMACS data 
directories as shown below:

\begin{tcolorbox}
\begin{Verbatim}[fontsize=\scriptsize]
EXECUTABLE=gmx 
SCRATCH=$MYSCRATCH/run_gromacs/$SLURM_JOBID
RESULTS=$MYGROUP/gromacs_results/$SLURM_JOBID
INPUT_DATA_DIR=${SLURM_SUBMIT_DIR}/gromacs_mpi
GROMACS_DATA_DIR=${SLURM_SUBMIT_DIR}
OUTFILE=1AKI_processed.gro
PDB_FILE=1AKI.pdb
\end{Verbatim}
\end{tcolorbox}

For this example, all of the gromacs and input data files were copied to the scratch.

\begin{tcolorbox}
\begin{Verbatim}[fontsize=\scriptsize] 
cp $GROMACS_DATA_DIR/*.pdb $SCRATCH
cp $INPUT_DATA_DIR/posre.itp $SCRATCH
cp $INPUT_DATA_DIR/1AKI_processed.gro $SCRATCH
cp $INPUT_DATA_DIR/topol.top $SCRATCH
cp $INPUT_DATA_DIR/*.mdp $SCRATCH
\end{Verbatim}
\end{tcolorbox}

Unlike the LAMMPS example, GROMACS consisted of many serial tasks which had to be performed first, before running the actual MPI task. These serial 
tasks were basically the pre-processing data for the MPI task and this was done as demonstrated below:
  
\begin{tcolorbox}
\begin{Verbatim}[fontsize=\scriptsize] 
aprun -n 1 -N 1 $EXECUTABLE pdb2gmx -f ${PDB_FILE} -o 
			$OUTFILE -water spce << EOF
29
0
EOF
aprun -n 1 -N 1 gmx_d editconf -f 1AKI_processed.gro -o 
	1AKI_newbox.gro -c -d 1.0 -bt cubic  > ${OUTPUT}

aprun -n 1 -N 1 gmx_d solvate -cp 1AKI_newbox.gro -cs 
spc216.gro -o 1AKI_solv.gro -p topol.top >> ${OUTPUT}

#Adding Ions
aprun -n 1 -N 1 gmx_d grompp -f ions.mdp -c 
1AKI_solv.gro -p topol.top -o ions.tpr >> ${OUTPUT}

aprun -n 1 -N 1 gmx_d genion -s ions.tpr -o 
1AKI_solv_ions.gro -p topol.top  -pname NA 
		-nname CL -nn 8 << EOF 
13
EOF

# Energy Minimization
aprun -n 1 -N 1 gmx_d grompp -f minim.mdp -c 
1AKI_solv_ions.gro -p topol.top -o 
		em.tpr >> ${OUTPUT}

aprun -n 48 -N 24 mdrun_mpi_d  -v -deffnm em -g 
			energy.log >> ${OUTPUT}
\end{Verbatim}
\end{tcolorbox}

The README script of GROMACS was very similar to the previous examples. However, the only difference was that it included many links to download the mdp
files that the SLURM script required for the input data sets. Therefore when ./README was typed, the README automatically downloaded these files with
the help of \emph{wget} as shown below:

\begin{tcolorbox}
\begin{Verbatim}[fontsize=\scriptsize]
wget http://www.bevanlab.biochem.vt.edu/Pages/
Personal/justin/gmx-tutorials/lysozyme/Files/ions.mdp

wget http://www.bevanlab.biochem.vt.edu/Pages/
Personal/justin/gmx-tutorials/lysozyme/Files/minim.mdp

wget http://www.bevanlab.biochem.vt.edu/Pages/
Personal/justin/gmx-tutorials/lysozyme/Files/nvt.mdp
\end{Verbatim}
\end{tcolorbox}


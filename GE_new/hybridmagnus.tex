\subsubsection{Hybrid Examples}

A hybrid job is a mixed job that is a combination of OpenMP and MPI tasks. It aims to take advantage of the OpenMP in NUMA region which is also known as 
a socket. The NUMA region was in this case involved one 12-core chip and ran on 6 threads on each of 8 MPI tasks which dispersed evenly between 
the NUMA regions.

A sum of 2 nodes was required which accomodated 8 MPI tasks and 6 threads. Besides specifying the number of nodes, the time was also shown in the slurm 
script as done in the previous examples. The partition used was also debugq for this task.

\begin{tcolorbox}
\begin{Verbatim}[fontsize=\scriptsize]
#SBATCH --partition=debugq
#SBATCH --nodes=2
#SBATCH --time=00:05:00
#SBATCH --export=NONE
\end{Verbatim}
\end{tcolorbox}


8 MPI tasks (-n 8) were specified to aprun to launch the job with 4 MPI tasks per node (-N 4), 2 MPI tasks per socket (-S 6) and each of the 8 MPI tasks 
had 6 OpenMP threads (-d 6). Therefore, the number of OpenMP threads, OMP\_NUM\_THREADS was set to 6.
This full expression is shown as:

\begin{tcolorbox}
\begin{Verbatim}[fontsize=\scriptsize]
export OMP_NUM_THREADS=6
aprun -n 8 -N 4 -S 2 -d 6 ./$EXECUTABLE >> ${OUTPUT}
\end{Verbatim}
\end{tcolorbox}

Once again, to run on multiple threads with the Intel environment on Cray, AFFINITY should be disabled with the same command used in the OpenMP examples
just after the aprun command.

The source codes used for the hybrid examples were also consisted of both Fortran and c which were hybrid\_hello.f90 and hello\_hybrid.c. The README 
script contained the compiling commands to compile on Cray as shown below:

\begin{tcolorbox}
\begin{Verbatim}[fontsize=\scriptsize]
ftn -O2 -h omp hybrid_hello.f90 -o hello_hybrid_cray
cc -O2 -h omp hybrid_hello.c -o hello_hybrid_cray
\end{Verbatim}
\end{tcolorbox}

To compiler with the GNU environment for fortran and c codes respectively, the README had the following compilers:

\begin{tcolorbox}
\begin{Verbatim}[fontsize=\scriptsize]
ftn -O2 -fopenmp hybrid_hello.f90 -o hello_hybrid_gnu
cc -O2 -fopenmp hybrid_hello.c -o hello_hybrid_gnu
\end{Verbatim}
\end{tcolorbox}

For the Intel environment, the compiler code changed to:

\begin{tcolorbox}
\begin{Verbatim}[fontsize=\scriptsize]
ftn -O2 -openmp hybrid_hello.f90 -o hello_hybrid_intel
cc -O2 -openmp hybrid_hello.c -o hello_hybrid_intel
\end{Verbatim}
\end{tcolorbox}

The same sbatch command as the OpenMP and MPI tasks was used for the hybrid codes to submit them to Magnus. The only difference was the name of the name
of the SLURM scripts.

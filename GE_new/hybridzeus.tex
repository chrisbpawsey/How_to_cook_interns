\subsubsection{Hybrid Examples}

The OpenMP/MPI hybrid codes used for Zeus were exactly same as the ones for Magnus such as FORTRAN and C code, but they both had
distinct compiler commands specific to each compiler module. Therefore, it was also necessary to include which module to be loaded in the README files
of the getexample tool.

This hybrid job requires 2 nodes and runs 1 MPI process with 16 OpenMP threads on each compiled executable. In order to launch the job to Zeus for both 
of the source codes, the number of OpenMP threads was set to 16 and the srun command was used.

\begin{tcolorbox}
\begin{Verbatim} [fontsize=\scriptsize]
export OMP_NUM_THREADS=16
srun --mpi=pmi2 -n 2 -N 2 ./$EXECUTABLE >> ${OUTPUT}
\end{Verbatim}
\end{tcolorbox}

This command was used for all of the SLURM files with different compilers without making any changes. To compile the hybrid\_hello.f90 and 
hello\_hybrid.c respectively with various compilers, the following commands were used:

For GNU:

\begin{tcolorbox}
\begin{Verbatim}[fontsize=\scriptsize]
mpif90 -O2 -fopenmp hybrid_hello.f90 -o hello_hybrid_gnu
mpicc -O2 -fopenmp -O2 hello_hybrid.c -o hello_hybrid_gnu
\end{Verbatim}
\end{tcolorbox}


For Intel:

\begin{tcolorbox}
\begin{Verbatim}[fontsize=\scriptsize]
mpif90 -O2 -qopenmp hybrid_hello.f90 -o hello_hybrid_intel
mpicc -O2 -qopenmp hello_hybrid.c -o hello_hybrid_intel
\end{Verbatim}
\end{tcolorbox}

For PGI:

\begin{tcolorbox}
\begin{Verbatim}[fontsize=\scriptsize]
pgf90 -Mmpi=sgimpi -mp hybrid_hello.f90 -o hello_hybrid_pgi
pgcc -Mmpi=sgimpi -mp hello_hybrid.c -o hello_hybrid_pgi
\end{Verbatim}
\end{tcolorbox}

\subsection{Magnus}

Magnus is described as a Cray X40 supercomputer that consists of many nodes that are bound by a high speed network. For the compute nodes, each one 
of them has 2 sockets and each of these has 12 cores. Magnus is also specified to have 24 cores per node and in total these sum up to 35,712 cores across 
the 1488 nodes. 

On Magnus, jobs run on the back-end of the system with the help of SLURM and ALPS (the Cray Application Level Placement Scheduler). A batch job is 
submitted to the queue system on the front-end from the sbatch command. When it runs, it executes the launch command \emph{aprun} on the MOM nodes 
which are the login nodes of Magnus. The \emph{aprun} keeps running on these login nodes until the application gets completed. Once \emph{aprun} finishes, the 
SLURM job also completes.

As mentioned previously, the focus of the \emph{getexample} tool was not only to provide working examples to the users, but also assist them on how to run jobs
on their scratch directories. Therefore, whenever the Batch job was submitted to Magnus, it was ran on the scratch directory and once the job was 
completed, the results were carried to the group directory. In the end, the scratch was removed. The examples used for Magnus were mainly parallel
programming examples such as MPI, OpenMP and hybrid codes which is a combination of OpenMP and MPI tasks and some applications such as LAMMPS and GROMACS.
The source codes used for these examples were written in C or FORTRAN which were basic \emph{"Hello world"} codes with the exception of the application codes.
Each example was displayed in different environments on Magnus such as GNU, Intel and more importantly the default environment, Cray with the compiler
options of \emph{ftn, mpif90} and \emph{cc.} When using these environments, it was important to load the actual programming environment before compiling the source
codes and running them as each environment has specific compiler commands with distinct wrappers.

\subsubsection{OMP Examples}

OpenMP (Open Multi-Processing) is known as a common shared memory model in which the threads work in parallel and access all shared memory. Therefore,
it can be thought of one task because of their restrictions to only one node. Unlike the MPI examples, the compiler commands and the wrappers do change 
for different environments on Magnus with the OpenMP tasks. Thus, it is necessary to swap to the right program environment corresponding to the compiler.

As mentioned previously, the OpenMP jobs use only one node and this node can be fully occupied node or a node occupying a single NUMA region. When it was 
fully occupied, the slurm script had a choice of only one node with the debugq partition as explained.

\begin{tcolorbox}
\begin{Verbatim}[fontsize=\scriptsize]
#SBATCH --partition=debugq
#SBATCH --nodes=1
#SBATCH --time=00:05:00
#SBATCH --export=NONE
\end{Verbatim}
\end{tcolorbox}

The initiation of the job by aprun is done by specifying that there is one task and 24 threads. The number of threads is also called a "depth" and hence,
the number of threads, OMP\_NUM\_THREADS was set to 24 giving an expression of:

\begin{tcolorbox}
\begin{Verbatim}[fontsize=\scriptsize]
export OMP_NUM_THREADS=24
aprun -n 1 -d 24 ./$EXECUTABLE >> ${OUTPUT}
\end{Verbatim}
\end{tcolorbox}

For a node occupying a single NUMA region, the code are more efficient when threads are limited to one NUMA region containing 12 cores. The expression 
(-d) 12 ensures that threads are bound correctly and this can be clearly done by specifying this via -cc which specifies the cores used. This is all 
illustrated below as:

\begin{tcolorbox}
\begin{Verbatim}[fontsize=\scriptsize]
export OMP_NUM_THREADS=12
aprun -n 1 -d 12 -cc 0-11 ./$EXECUTABLE >> ${OUTPUT}
\end{Verbatim}
\end{tcolorbox}

The source codes used for OpenMP jobs were omp\_hello.f and omp\_hello.c. To run these source codes respectively on Cray, the README included the 
following compiling commands:

\begin{tcolorbox}
\begin{Verbatim}[fontsize=\scriptsize]
ftn -O2 -h omp omp_hello.f -o hello_omp_cray
cc -O2 -h omp omp_hello.c -o hello_omp_cray
\end{Verbatim}
\end{tcolorbox}

Then the job was also submitted to Magnus by using the same sbatch command in the README file of the MPI examples except replacing the name of the SLURM
with the correct name for the OpenMP task.

To compile with the GNU environment, the compiler for Fortran and c codes respectively changed to: 

\begin{tcolorbox}
\begin{Verbatim}[fontsize=\scriptsize]
ftn -O2 -fopenmp omp_hello.f -o omp_hello_gnu
cc -O2 -fopenmp omp_hello.c -o omp_hello_gnu
\end{Verbatim}
\end{tcolorbox}

To run the OpenMP code correctly on Cray with the Intel environment, the affinity should be disabled. If it is not disabled, the code runs on only one
thread rather than running on multiple threads and this was done by adding the following command before the aprun:

\begin{tcolorbox}
\begin{Verbatim}[fontsize=\scriptsize]
export KMP=AFFINITY=disabled
\end{Verbatim}
\end{tcolorbox}

For the Intel environment, the compilers used were expressed as:

\begin{tcolorbox}
\begin{Verbatim}[fontsize=\scriptsize]
ftn -O2 -openmp omp_hello.f -o omp_hello_intel
cc -O2 -openmp omp_hello.c -o omp_hello_intel
\end{Verbatim}
\end{tcolorbox}

\subsubsection{OMP Examples}

The OpenMP examples used for Zeus had the same source codes as the ones used for Magnus which were \emph{omp\_hello.c} and \emph{omp\_hello.f}
However, as mentioned earlier due to the different setups between Magnus and Zeus, the SLURM and the README for these codes are different.

The main difference between the MPI and OpenMP jobs for Zeus is that the compiling command changes for all the different compiler options with OpenMP
commands as the wrappers are not the same for GNU, Intel and PGI compilers. It is vital to notice that the commands to compile \emph{hello\_mpi.f90} and
\emph{hello\_mpi.c} were the same for GNU and Intel compilers for the MPI jobs, whereas this is not the same case for OpenMP jobs anymore. However, 
everytime the compiler is swapped from one to the other, the correct compiler module should be loaded because the default compiler module is gcc. If not, 
the system will fail to recognize the compiler commands and end up giving many errors, even if the source codes and the SLURM files work.

In the \emph{getexample} tool, to run an OpenMP job on Zeus, only one node was used just as Magnus, but as diccussed in the MPI example, the \emph{workq} 
partition was used unlike the \emph{debugq} partition. Therefore, both of the OpenMP examples ran one 16-thread OpenMP instance with one node.

To run the \emph{omp\_hello.f} and \emph{omp\_hello.c} codes on Zeus with the GNU compiler, the number of OpenMP threads was set to 16 on the SLURM and 
the srun command was used to run it as shown:

\begin{tcolorbox}
\begin{Verbatim}[fontsize=\scriptsize]
export OMP_NUM_THREADS=16
srun -n 1 -c $OMP_NUM_THREADS ./$EXECUTABLE >> ${OUTPUT}
\end{Verbatim}
\end{tcolorbox}

An alternative way of running the same job could be using the omplace command.

\begin{tcolorbox}
\begin{Verbatim}[fontsize=\scriptsize]
omplace -nt $OMP_NUM_THREADS -tm open64 ./$EXECUTABLE 
                                         >> ${OUTPUT}
\end{Verbatim}
\end{tcolorbox}

The srun command above can be used for both Intel and PGI compilers without requiring modification. However, the omplace command would be different for
the other compilers. It is essential to note that -tm open64 should be included above in the omplace command because when compiling with GNU, this 
command will not be identified, as the default thread model is intel. Hence, -tm open64 tells that the compiler module is GNU. Therefore, to run this job
with the Intel compiler, the omplace command in the SLURM would look like:

\begin{tcolorbox}
\begin{Verbatim}[fontsize=\scriptsize]
omplace -nt $OMP_NUM_THREADS ./$EXECUTABLE >> ${OUTPUT}
\end{Verbatim}
\end{tcolorbox}

To compile the \emph{omp\_hello.f} with the GNU compiler, the following command was used in the README file:

\begin{tcolorbox}
\begin{Verbatim}[fontsize=\scriptsize]
gfortran -O2 -fopenmp omp_hello.f -o hello_omp_gnu
\end{Verbatim}
\end{tcolorbox}

The \emph{omp\_hello.c} code was compiled with the GNU as shown:

\begin{tcolorbox}
\begin{Verbatim}[fontsize=\scriptsize]
gcc -O2 -fopenmp omp_hello.c -o hello_omp_gnu
\end{Verbatim}
\end{tcolorbox}

As it can be seen, the only difference between compiling the C code and the FORTRAN code was the gcc and gfortran commands. Both of these codes can be 
compiled with Intel and PGI compilers, but the wrappers for them are different.

For Intel, compiling the FORTRAN and C codes respectively:

\begin{tcolorbox}
\begin{Verbatim}[fontsize=\scriptsize]
ifort -O2 -qopenmp omp_hello.f -o hello_omp_intel
icc -O2 -qopenmp omp_hello.c -o hello_omp_intel
\end{Verbatim}
\end{tcolorbox}

For PGI, compiling the FORTRAN and C codes respectively:

\begin{tcolorbox}
\begin{Verbatim}[fontsize=\scriptsize]
pgfortran -O2 -mp omp_hello.f -o hello_omp_pgi
pgcc -mp omp_hello.c -o hello_omp_pgi
\end{Verbatim}
\end{tcolorbox}

\subsubsection{OMP Examples}

For the OpenMP examples, the source codes used were \emph{omp\_hello.f90} and \emph{omp\_hello.c} which were compiled with GNU, Intel and PGI 
that used similar compiling commands as the OpenMP examples of Zeus. The SLURM script for the OpenMP task requests 2 nodes in comparison to the MPI tasks 
and 6 cores per node with 128 GB of memory each which gives a total of 12 cores. As mentioned previoulsy in the MPI examples, the 2 nodes were specified 
as \emph{--ntasks} rather than \emph{--nodes} in the SLURM directives as shown:

\begin{tcolorbox}
\begin{Verbatim}[fontsize=\scriptsize]
#SBATCH --partition=zythos
#SBATCH --ntasks=2
#SBATCH --cpus-per-task=6
#SBATCH --account=pawsey0001
#SBATCH --time=00:10:00
\end{Verbatim}
\end{tcolorbox}

To run the job on Zythos, omplace was used for the OpenMP to control thread placement with a default of 6 threads per node which gave a total of 12 
threads. This was written in the SLURM as:

\begin{tcolorbox}
\begin{Verbatim}[fontsize=\scriptsize]
export OMP_NUM_THREADS=12
omplace -nt $OMP_NUM_THREADS ./$EXECUTABLE >> ${OUTPUT}
\end{Verbatim}
\end{tcolorbox}

Since Zythos is a node of Zeus, to compile the source codes and submitting the SLURM script to the Zythos, the same commands as Zeus were used and thus,
the README and the SLURM scripts for the OpenMP jobs were similar to ones of Zeus.


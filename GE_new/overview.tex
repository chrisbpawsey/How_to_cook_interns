
\section{Overview of getexample}

The main aim of the "getexample" is to provide easy access to the examples for the users. For this reason, the getexample tool is expected to be have the 
following design specifications:

\begin{itemize}
\item It should be easy to use for everyone including beginners. The user should be able to download the examples and run them with one command.
\item It should provide practical examples which can be modified or updated by other users for their own work. Therefore, it encourages the users to learn
how to use the resources given to them effectively.
\item The example should be clear and completely detailed with steps to help the users understand and apply it.
\end{itemize}

The getexample tool is designed in a way such that when "getexample" command is typed from any directory on the resources of Pawsey Centre, it lists all 
the examples within the getexample library. As the getexample displays many examples which aim to perform various tasks for each individual resource at
Pawsey, it ensures that when the user logins from a specific supercomputer such as Magnus, it only lists the examples provided for Magnus rather than
showing all of the examples within the library. To obtain a copy of any example, the user simply types "getexample" followed by the name of the example 
as shown below:

\begin{tcolorbox}
\begin{Verbatim}[fontsize=\scriptsize]
getexample <name of the example>
\end{Verbatim}
\end{tcolorbox}

This creates a new directory with same name as the example requested in wherever the getexample tool is accessed from and it downloads all the files of
the example to the new directory created. The new directory then can be accessed by the user simply typing:

\begin{tcolorbox}
\begin{Verbatim}[fontsize=\scriptsize]
cd <name of the example>
\end{Verbatim}
\end{tcolorbox}

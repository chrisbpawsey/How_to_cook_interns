\section{How to Implement the Getexample Command}

Due to the different operating systems of the Pawsey supercomputers, each individual supercomputer had seperate examples that could display the same
function, but with different commands. Therefore when implementing the getexample command, the key principal was to ensure that it only displayed the
examples related to the supercomputer in use. For example, if the user logins from Zeus, when one types the command \emph{getexample}, it should only
show the examples that can be performed on Zeus. This prevents the users from forcing to run examples that are not applicable for that particular 
supercomputer. For example, if the user logins from Magnus and tries to run an example taken from Zeus that compiles with GNU, on Magnus with the GNU 
environment the code will fail to run. Even though the program environment on Magnus was set to GNU, the compiler commands as well as the commands to
run the task on Magnus differ from the ones of Zeus.

In order to follow this design specification, the getexample command was developed by using the following Bash script that determines which examples to
list based on the \emph{hostid} of the supercomputer.    


\begin{tcolorbox}
\begin{Verbatim}[fontsize=\scriptsize]
#!/bin/bash
#DESCRIPTION Download user examples
#LABEL Files
hostid=`echo $HOST | cut -d "." -f 1`
echo ${hostid}
if [ ${hostid} == "magnus" ]; then 
  location=/group/interns2016/getexample/magnus_examples
elif [ ${hostid} == "zeus" ]; then
  location=/group/interns2016/getexample/zeus_examples
fi

#TODO Make a list of users downloading examples?

#Display help message if no example is given
if [ "$1" == "" ]
then
    echo "Download an HPC example:"
    echo "usage:"
    echo "   getexample <examplename>"
    echo ""
    echo "Where <examplename> is the name of the example 
    you want to "
    echo "download.  This will create a directory 
    named examplename which"
    echo "you can cd into and hopefully read the 
    README file (if one is"
    echo "avaliable) or just submit the *.qsub file."
    echo ""
    echo "For Example:"
    echo "  getexample helloworld"
    echo ""
    echo "Possible example names:"
    ls $location 
    exit 0
fi

#TODO Check to see if folder already exists
cp -r -v -u ${location}/$1 .
\end{Verbatim}
\end{tcolorbox}

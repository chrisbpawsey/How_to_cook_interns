\subsection{Zeus}

According to Pawsey system descriptions, Zeus is an SGI Linux cluster that is principally utilized for pre and post-preparing of data, extensive shared 
memory calculations and remote visualization work. It shares the same /home,/scratch and /group file systems with other Pawsey Centre systems such as 
Magnus. Zeus is a heterogeneous cluster with a large number of back-end nodes with both CPU an GPU hardware which makes Zeus an excellent resource to run
jobs like CUDA codes. To access the back-end nodes, a SLURM script is used just like Magnus except the default partition queue is workq instead of 
debugq. There are other nodes on Zeus such as copyq which is another partition queue that can be only used for specific purposes. Zeus also contains one 
large-memory node known as Zythos, which is an SGI UV2000 system and Zythos can be accessed from the front end login node of
Zeus. 

The getexample tool also provided some examples for Zeus which were also mainly based on parallel programming tasks. The SLURM and README scripts
contained the same design as the ones for Magnus with only differing commands for running the task on Zeus, compiling the source codes and loading the
correct compiler modules. 
 

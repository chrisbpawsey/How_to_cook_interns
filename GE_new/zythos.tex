\subsection{Zythos}
As mentioned earlier, Zythos is a large-memory node of Zeus and can be accessed through the front end login node of Zeus. Unlike Zeus, Zythos is a 
restricted asset and liable to a strict qualification criteria. One of the following criteria must be met for the task to be performed on Zythos:

\begin{itemize}
\item A vast detailed data should be held in shared memory, and be larger than 512GB.
\item Gigantic thread-level parallelism, for example, utilizing a large number of CPU cores.
\end{itemize}

It is more preferrable that the work meets both criteria. If the work does not satisfy either one of these criteria, then Zeus is a better source to
use. In comparison to  Magnus and Zeus, there is no direct access to Zythos, all work is facillitated via Zeus. However, the partition queue is no longer
the workq and it changes to zythos. Additionally, the codes on Zythos does not run properly if the account name is not specified or 
entered correctly as specific accounts are authorized to run on Zythos. For this reason, it is necessary to change account name and put the authorized 
account, otherwise the job does not run. Therefore, this brings out the challange of having to modify the account name in the SLURM scripts of the 
\emph{getexample} tool which means that the \emph{getexample} is not completely automated for Zythos. The users who are interested in running these 
examples on this resource will not be simply typing ./README to execute the example but instead will have to replace the account name entered in the 
examples with their own. The examples provided for Zythos were also parallel programming codes such as MPI, OpenMP and hybrid tasks and these jobs were 
also run with GNU, Intel and PGI compilers which utilized exactly the same compiler commands as the Zeus examples. However, to run the jobs on Zythos, 
the SLURM scripts included different commands from srun such as omplace and mpirun. 


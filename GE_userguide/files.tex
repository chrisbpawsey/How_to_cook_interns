\section{ Files Included in the Getexample}

Each individual example on the \emph{getexample} occupies a directory that is reserved for them with consisting of two files listed below:

\begin{itemize}
\item SLURM (Simple Linux Utility for Resource Management): This allows the users to submit batch jobs to the supercomputers, check on their status and
cancel them if needed. It contains the necessary information about the name of the executable, the results directory, the name of the output file and
the jobID. It also allows the users to edit how many nodes the code requires to run on Magnus and Zeus, the duration of the task that it takes, which
partition to be used and their account name. The SLURM initially creates a scratch directory for the example to run in and the results are outputted to
a log file. Then, it creates a group directory in which the results directory is located for that example. Once the job finishes running within the
scratch, the output file is then moved to the results directory located in the group directory and then the scratch directory gets removed.
\item README: This file is an executable Bash script which can be read and run by the users. It provides the path to the source code and includes 
details about what the source code does, how to compile the source code depending on the program environment such as Cray, Intel, GNU or compilers like 
PGI, what can be modified in the SLURM directives, and a set of instructions on how to submit the SLURM to Magnus and Zeus including which specific 
commands to use for that particular supercomputer. It can be executed by simply typing ./README which then uses the path to copy the source code into the 
directory of the example, then it compiles the source code and submits the batch job to the chosen supercomputer.
\end{itemize}



\section{Further Information}

The \emph{getexample} tool produced by Pawsey Supercomputing Centre is mainly implemented to make the introduction to supercomputing easier for the new 
users to help them understand how to run serial and parallel computing codes on Magnus and Zeus using the commands specific to each individual
supercomputer. It also tries to instil into the users the behaviour of running jobs on the scratch directory, but storing the results in the group directory
instead.

Therefore, Pawsey Supercomputing Centre hopes that this userguide is not only helpful and beneficial for the beginners, but also for the other users. 
Should you require further information on how to run jobs on Magnus and Zeus including how many nodes to specify for a particular task and which
compilers to use, do not hesitate to visit Pawsey Supercomputing Centre's website. The links to Magnus and Zeus user guides are provided below:

\begin{itemize}
\item For Magnus:
\begin{enumerate}
\item \url{https://support.pawsey.org.au/documentation/display/US/Compiling+on+a+Cray}
\item \url{https://support.pawsey.org.au/documentation/display/US/Running+on+a+Cray}
\end{enumerate}
\item For Zeus:\\
\url{https://support.pawsey.org.au/documentation/pages/viewpage.action?pageId=2162999}
\end{itemize}


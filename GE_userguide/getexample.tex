\section{How to Use the \emph{Getexample} on Pawsey Supercomputing Centre Resources}

The \emph{getexample} tool currently provides examples for Magnus and Zeus which are easily accessible from both of the resources. However, it provides
examples that are applicable to the supercomputer in use. Therefore, if you login from Magnus, you can only display the Magnus examples. To view the
Zeus examples, you need to login with your Zeus account.

To use the \emph{getexample}, follow the steps listed below:

\begin{enumerate}
\item To have access to the examples, login from either Magnus or Zeus, using the following commands respectively.
\begin{tcolorbox}
\begin{Verbatim}[fontsize=\scriptsize]
ssh -X myusername@magnus.pawsey.org.au
ssh -X myusername@zeus.pawsey.org.au
\end{Verbatim}
\end{tcolorbox}
If you are using a Windows machine, you need to use ssh -X (a relaxed security mode). If you are a Mac user, you need to specify ssh -Y instead.
\item To view the examples within the \emph{getexample}, simply type:
\begin{tcolorbox}
\begin{Verbatim}[fontsize=\scriptsize]
getexample
\end{Verbatim}
\end{tcolorbox} 
This lists all the examples available for the supercomputer you are utilising.
\item To require one of the examples, simply type: 
\begin{tcolorbox}
\begin{Verbatim}[fontsize=\scriptsize]
getexample <name of the example>
\end{Verbatim}
\end{tcolorbox} 
This creates a directory with the same as the example on where you access the \emph{getexample} from and copies the files of the example into this 
directory.
\item To see the example, change your directory from your current directory to the new directory created by typing: 
\begin{tcolorbox}
\begin{Verbatim}[fontsize=\scriptsize]
cd <name of the example>
\end{Verbatim}
\end{tcolorbox}
\item To list the files located in this example directory, type: 
\begin{tcolorbox}
\begin{Verbatim}[fontsize=\scriptsize]
ls
\end{Verbatim}
\end{tcolorbox}
\item To run the example, simply type: 
\begin{tcolorbox}
\begin{Verbatim}[fontsize=\scriptsize]
./README
\end{Verbatim}
\end{tcolorbox}
To understand what the README does, we recommend you to read the descriptions located with the README script.
\end{enumerate}



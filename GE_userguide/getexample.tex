\section{How to Use the \emph{Getexample} on Pawsey Supercomputing Centre Resources}

The \emph{getexample} tool currently provides examples for Magnus and Zeus which are easily accessible from both of the resources. However, it provides
examples that are applicable to the supercomputer in use. Therefore, if you login from Magnus, you can only display the Magnus examples. To view the
Zeus examples, you need to login with your Zeus account.

To use the \emph{getexample}, follow the steps listed below:

\begin{enumerate}
\item Login from either Magnus or Zeus. To login to Magnus or Zeus, use the following commands respectively. If you are using a Windows machine, you
need to use ssh -X. If you are a Mac user, you need to specify ssh -Y instead.  
\begin{enumerate}
\item ssh -X myusername@magnus.pawsey.org.au
\item ssh -X myusername@zeus.pawsey.org.au
\end{enumerate}
\item To access the examples within the \emph{getexample}, simply type: getexample. This lists all the examples available for the supercomputer you are
utilising.
\item To require one of the examples, simply type: getexample \<name of the example\>. This creates a directory with the same as the example on where you
access the getexample from and copies the files of the example into this directory.
\item To view the example, change your directory from your current directory to the new directory created. To do this, type: cd \<name of the example\>
\item To list the files located in this example directory, type: ls
\item If you wish to run the example, you need to use: ./README.
\end{enumerate}



\section{Getexample Outline: What Is Getexample and Why Is It Created?}

\emph{Getexample} is a tool designed for Pawsey Supercomputing Centre resources which provides working examples for serial and parallel programming for
the users of Pawsey. 

As Pawsey Supercomputing Centre has many resources which are set up differently from each other, it is usually difficult for the users to find sample 
codes which could be run on Pawsey's supercomputers without any modification. Even if the users manage to find well-written Bash scripts, they may not 
know how to submit these examples to the supercomputers they wish to use. Therefore, the \emph{getexample} displays various examples to perform specific 
tasks for two particular supercomputers of Pawsey which are Magnus and Zeus. Hence, it aims to help the users in many areas listed below:

\begin{itemize}
\item It provides serial and parallel programming examples such as MPI, OpenMP and hybrid jobs which is a combination of OpenMP and MPI. Hence, it 
demonstrates the users what the main differences are between these examples and shows how to run them with various compilers and program environments.
\item It supplies information about how many nodes are required for different tasks and which particular commands to utilize to submit them to 
Magnus and Zeus such as \emph{aprun, srun} or \emph{omplace}. 
\item It shows how to specify the partition queue and runs all the jobs on the \emph{workq}.
\item It displays how to run tasks on the scratch and store the results in the group directories effectively.
\end{itemize}   

Thus, the \emph{getexample} hopes to be a self-directed learning tool, helping the users develop and perform their own projects utilizing the applicable
commands from the \emph{getexample} based on their preferences. 

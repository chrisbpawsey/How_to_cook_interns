% begin the document
\begin{Document}

\Title{Internship Report for Ranganai Mapondera and Ozlem Erkilic}
\Subtitle{Getexample: Reducing Barriers to Entry on Shared HPC Resources}
\Section{Abstract}
\begin{abstract}




\end{abstract}
\\
\Section{Introduction}
\begin{Introduction}




\end{Introduction}
\
\
\\
\Section{Scope}
\begin{scope}

This shows the expected results from this Internship at Pawsey Supercomputing Center
\
\



\end{scope}
\
\\
\Section{Magnus}

This is described as a Cray X40 supercomputer that consists of many nodes that are connected by a high speed network. The cray supercomputer consists of intel Xeon E5-2690 v3(Haswell) as its only processor across all compute nodes. For the compute nodes, each one of them has 2 processors and each of these 2 processors has 12 cores.
Magnus is specified to have 24 cores per node and in total these sum up to 35,712 cores across the 1488 nodes. Each of these nodes has a memory of 64GB. The maximum login nodes of magnus are 2.
During this summer internship, the focus was to provide simplified getexamples which we developed and each othe directories described below consists of mainly the source code, the slurm file and the Readme . The source code details the script with the example code taken from online sources or developed to be the job script. The jobscript produces the executable file after being compiled. The slurm file is where a batch script is submitted. It runs the executable from the jobscript. The executable file will be used by the compiler. In the slurm, we have slurm directives where we choose the number of nodes or ntasks. We also specify the compiler before the executable. The Readme file is where the executable file is run. The slurm is given because it contains the compiler. The results from running the executable are submitted to Scratch which deletes the files after some time. A new group has been created which stores all the files from scratch and in the script the results are deleted from scratch and only viewed in group. Compute nodes are attached to scratch and are capable of moving back and forth.
Fortan and C code were used for most of the examples to illustrate the differences. Some wrappers were implemented as well for these compilers which are ftn,mpif90 and cc. When using these, it is important to invoke the actual programming environment being used PrgEnv-cray, PrgEnv-gnu or PrgEnv-intel.The relevant paths are organized via the module system.



\Subsection{MPI Jobs}

For MPI , we used fortran and c compilers. We ran the jobs on cray, gnu and intel.


\Subsection{OMP Jobs}



\Subsection{Helloworld Jobs}



\Subsection{CUDA Jobs}



\Subsection(Hybrid Jobs}


\Section{Zeus}



\Subsection{MPI Jobs}




\Subsection{OMP Jobs}



\Subsection{Helloworld}



\Subsection{Cuda Jobs}



\Subsection{Hybrid Jobs}


\Section{Zythos}



\Subsection{MPI Jobs}



\Subsection{OMP Jobs}



\Subsection{Helloworld}



\Subsection{Cuda Jobs}



\Subsection{Hybrid Jobs}


\Section{Other Completed Tasks}

\Subsection{Intel Xeon Processor}

\Subsection{MIC &MKL}


\Susection{Ubuntu}



\Subsection{Supercomputing Examples to Others}



\Section{Conclusion}



\Section{References}
% end document

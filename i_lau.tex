\section{Sarawak Laksa}
A treat from the islands of Borneo, Malaysia – the traditional and authentic Sarawak laksa is a noodle soup that is cooked with a difference. Although the ingredients used are very similar to that of other laksa variants, the uniqueness lies in the method of cooking it.

\subsection{Ingredients}
\begin{itemize}
	\item Rice vermicelli
	\item Prawns with the head (save the shells for the broth)
	\item Chicken breasts
	\item Omelette
	\item Sprouts
	\item Coconut milk
	\item Chicken stalk
	\item Onions
	\item Galangal
	\item Sambal belacan paste
	\item Salt and sugar for seasoning
\end{itemize}

\subsection{Methods}
\begin{enumerate}
\item Boil chicken in water till it is well-cooked. Then separate the stock and the chicken. Separate the bones from the chicken meat and shred the chicken into small pieces.
\item Saute the prawn shells and blitz it with some chicken stock and the chicken bones. Strain the mixture and add it to the remaining chicken stock. Bring it all to a boil.
\item To the chicken and prawn shell stock add the coconut milk, Sambal Belacan paste, sugar and salt and bring them to a boil.
\item While the broth is cooking, blanch the rice noodles, prawn and sprouts. Also, make an omelette and cut it into thin strips.
\item To finish, in a bowl place the cooked rice noodles, shredded chicken, prawns and sprouts. Add the spicy broth and omelette strips. Garnish with fresh coriander leaves and chopped chillies.
\end{enumerate}

